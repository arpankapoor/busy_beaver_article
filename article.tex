\documentclass[a4paper]{article}
\usepackage{algpseudocode}
\usepackage{amsmath}
\usepackage{amssymb}
\usepackage[backend=biber]{biblatex}
\usepackage{geometry}
\usepackage{hyperref}
\usepackage[utf8]{inputenc}

\addbibresource{references.bib}
\hypersetup{
  pdftitle={The Busy Beaver Problem},
  pdfauthor={Deepak Sirone J, Arpan Kapoor}
}

\begin{document}
\title{The Busy Beaver Problem}
\author{Deepak Sirone J \and Arpan Kapoor}
\maketitle

A function is merely an abstraction used by mathematicians. It is defined by
its input type called the domain, output type called the range and suitable
definitions of the domain and the range. A suitable definition is one which
does not cause ambiguity, that is one which does not render itself to multiple
interpretations.

For example:
\begin{enumerate}
  \item \(f(x) = x \times x\) , where \(x \in \mathbb{R}\).
    Here we assume that the operation \(\times\) is closed in \(\mathbb{R}\)

  \item \(f(x) = \{e \mid \forall k \in x,\space \space e \geq k\}\) where
    \(x\) is a finite subset of natural numbers
\end{enumerate}

Many a time we are interested in computing the output of a function using an
algorithm. The description of the first example in this case, implicitly gives
us an algorithm for computing the output if the algorithm for performing the
operation \(\times\) is known. The second example intuitively asks us to
iterate through the elements in the set and find the largest element in it.

However, the notion of a function does not define the algorithm to get from
the input to the output. This means that using this definition of functions,
you have the freedom to define your own functions, provided you can specify the
domain and range precisely. So a function which outputs the number of sand
grains in the Sahara desert at a particular instant of time is perfectly legal!
This however does not ensure that there exists an algorithm to do it.

The example functions above, are definable by an algorithm. Such functions are
called computable functions. There also exists functions which can be proved
to not have an algorithm and they are called non-computable functions. One such
function known as the Busy Beaver function was defined by Timos Rado in his
1961 paper ``On Non-Computable Functions''\cite{rado_non-computable_1962}.

The busy beaver function is concerned with a model of executing algorithms known
as Turing machines. A Turing machine is a machine which has an infinite
bidirectional input/output tape, a tape head and rules governing the movement
of the tape head on the tape. The tape is divided into cells and the symbols
written on the tape are from an alphabet \(\Sigma\). Similar to a finite state
machine, a Turing machine has a set of states, \(Q\) and a transition function
\(\delta\). The tape head reads one symbol on the tape at a time and based on
the transition function, overwrites the symbol with another symbol and shifts
(steps either right or left on the tape
\footnote{The Turing machine which we define here does not have a center shift,
  meaning that the tape head must move left or right after each move}. The
Turing machine also has a halt state whereby the machine stops running.
The running time of a Turing machine on an input is measured as the number of
shifts it performs before it enters its halting configuration. A binary Turing
machine is one with \(\Sigma = \{0 , 1\}\).

Not all Turing machines stop running on its given input. The problem of
determining whether a Turing machine \(M\) halts on input \(x\) is known as the
halting problem. The halting problem has been shown to be undecidable,
meaning that there is no Turing machine (or algorithm for that matter) to solve
it. We could try simulating a given machine, to see if it stops after a while.
But the problem is that we don't know for how long the simulation should
continue. Suppose we simulate a Turing machine \(M\) on input \(x\) using
another Turing machine \footnote{called the Universal Turing Machine or UTM}
for \(t\) steps and conclude that it does not halt, we can always find another
Turing machine \(M'\) which halts on input \(x\) in \(t + 1\) steps.
Hence simulation does not work.

Turns out that the Turing machine model can simulate circuits (both sequential
and combinational) and hence we can safely conclude that it is capable of
simulating all algorithms that can be run on a real computer.
% Turing machines are useful for proving things when compared to your laptop,
% as they are mathematically better defined, and relatively simple.

%A \(k\)-state Turing Machine is a 7 tuple M = \((Q, \Sigma, \tau, \delta, q_0,
%  q_\text{accept}, q_\text{reject})\), where
%\begin{enumerate} don't think we need this 
%  \item \(\Sigma\) is the input alphabet,
%  \item \(\tau\) is the tape alphabet, where \(\Sigma \subseteq \tau\),
%  \item \(\delta: Q \times \tau \rightarrow Q \times \tau \times \{L, R\}\),
%  \item \(q_0 \in Q\) is the start state,
%  \item \(q_\text{accept} \in Q\) is the accept state,
%  \item \(q_\text{reject} \in Q\) is the reject state.
%\end{enumerate}

We call a binary Turing Machine a \(k\)-state Busy Beaver if on an empty tape as
input it halts and runs for at least as many steps as any other \(k\)-state
binary Turing machine.

In order to find a \(k\)-state Busy Beaver, we would have to evaluate all
\(k\)-state binary Turing machines. How many such Turing machines exist?
Since we have fixed the cardinality of Q and the alphabet
(\(\Sigma = \{0, 1\}\)), the number of possible Turing machines is equal to
the number of possible transition function. The transition function's domain is
the cross product of the set of states with the alphabet and codomain is the
cross product of the set of states, the alphabet and the set \(\{L, R\}\).

\[\delta: Q \times \Sigma \rightarrow Q \times \Sigma \times \{L, R\}\]
\[|Q| = k\]
\[|\Sigma| = 2\]
\[|\{L, R\}| = 2\]

From the above, we see that there are \({(2 \times 2 \times k)}^{2k}\), i.e.
\({(4k)}^{2k}\) number of \(k\)-state binary Turing machines.

The Busy Beaver function's input is a positive integer \(n\) and it outputs the
number of steps taken by a \(n\)-state Busy Beaver Turing machine to halt.
Assuming this function to be computable gives us the following algorithm to
solve the halting problem, which we know is not possible. Hence our assumption
must be wrong.

\begin{algorithmic}
  \Function{halts?}{M, x}
  \State k = \(|\)M.Q\(|\) \Comment{Number of states of the Turing machine M}
  \State bbk = \Call{BB}{k} \Comment{Number of steps taken by a \(k\)-state Busy Beaver}
  \If{\Call{UTM}{M, x, bbk}}
  \Return \textsc{Halts} \Comment{UTM simulates M on input x for bbk steps}
  \Else
  \space \Return \textsc{Does NOT Halt}
\EndIf
\EndFunction
\end{algorithmic}

Another interesting fact that can be observed using the same argument is that
the Busy Beaver function grows faster than all computable functions,
i.e. \(f(n) = O(BB(n))\) for any computable function \(f\). This is because if
some computable function \(f\) grew faster than the Busy Beaver function, the
function call to BB in the above algorithm can be replaced with a call to this
function \(f\), enabling us to solve the halting problem. Again our assumption
would be wrong. Table (xx) shows what we know currently about the
busy beaver series.

\begin{center}
  \begin{tabular}{ |l|l|l|l|l|l| }
    \hline
    & 2 State & 3 State & 4 State & 5 State & 6 State \\ \hline

    \(BB\) & 6 & 21 & 107 & \(\geq 47,176,870\) & \(> 7.4 \times 10^{36534}\)
    \\ \hline
  \end{tabular}
\end{center}



So far only we know the exact values of the busy beaver series upto 5 state
Turing machines. The Busy Beaver series suggests a new method for solving
theoretical problems in mathematics. For example, suppose we have a Turing
machine \(M\) which verifies Goldbach's conjecture (which states that every
even number n, \(n \geq 4\) is the sum of two primes) and it has \(k\) states.
If we were able to find \(BB(k)\) exactly then we could run \(M\) for \(BB(k)\)
steps and conclude that the conjecture holds if it does not stop within that
many steps, else we conclude that the conjecture does not hold.

\nocite{*}
\printbibliography
\end{document}
